\thispagestyle{empty}
\begin{center}\section*{Abstract}\end{center}
Power series is of great importance in solving differential equations, with results that can be used as a basis both for the representation of functions, primarily special functions, such as for use in various kinds of problems. This work is related to the heat diffusion process through an infinite cylinder, shaped by a partial differential equation. This equation, called the \textbf{Diffusion Equation}. The resolution established here proposes to solve this equation in its particular formulation in case of an \textbf{infinite cylinder} with racial distribution and prescribed temperature. Given especially via separation method in the following manner $ t = \vartheta \exp[-a k^2 \tau], $ where $ \vartheta $ is the solution of the differential equation $ \nabla^2 \vartheta + k^2 \vartheta = 0 $. In our case $ \vartheta (x) $ is a solution of the \textbf{Bessel equation}, is through power series about an ordinary point or a regular singular point. Initially, aiming at applying the Frobenius method, an extension of the method of the power series, conditions are established so that the ordinary point is characterized as a removable regular singular point, so that the Frobenius method also produces the so-called analytical solutions about an ordinary point, possibly by a series multiplying the logarithmic power term or by a fractional exponent. Then, using their respective recurrence relationship of the Euler-Cauchy equation, the two solutions are determined EDO. The analysis of linear dependence or independence of such solutions is made by means of Theorem General Solution of Bessel Equation.
\\
\\
\textbf{Keywords:} Diffusion equation. Bessel equation. Infinite cylinder. Heat transfer.

